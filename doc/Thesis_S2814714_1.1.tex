\documentclass[12pt,a4paper]{article}
\usepackage[latin2]{inputenc}
\usepackage{graphicx}
\usepackage{ulem}
\usepackage{amsmath}
\begin{document}
\begin{center}PAN 2016 shared task \end{center}

\begin{center}Author identification\end{center}

\begin{center}Can a clustering sandwich help a neural network
\end{center}

\begin{center}in establishing authorship links?\end{center}



\begin{center}Olivier Louwaars (s2814714), Department of Information 
Science\end{center}

\begin{center}Rijksuniversiteit Groningen\end{center}



\newpage


Abstract

Faced with the problem of having an unknown number of authors writing an 
unknown number of documents that might but do not necessarily have the 
same topic, we came up with a pipeline of a K-Means clustering algorithm 
informing a neural network to establish document similarity and which 
then informs a Meanshift algorithm that outputs the final clusters (each 
cluster representing an author and enveloping one or more documents). 
The data provided by PAN consists of 18 problems that each consist of 
50-100 documents from various authors. Every problem has either news 
articles or reviews, and is written in English, Dutch or Greek. Document 
length differs from 130 to 1000 words. This is not enough training data 
for a neural network, so character based features were created for 
training. Also, the task itself limited feature selection to 
character-only as well. If words or word n-grams would have been used, 
the initial assumption was that the system would be tricked easily into 
clustering on topic instead of author. As all documents within a problem 
have one genre, it is not unimaginable that documents from different 
authors have the same topic, and are therefore grouped together. 
Preventing the system from topic clustering and aiming it at author 
clustering was one of the biggest challenges in this task.\\
The most promising feature for this task were character based skipgrams 
$[$1$]$ pairing every character with either a neighboring one (positive 
sample) or one further away (negative sample). The embeddings thus 
created informed the neural network with underlying information 
regarding character sequences and structures. Although the task is to 
cluster all documents of a single author, the chosen approach was to 
train the neural network with every possible document pair, like the PAN 
2015 task for author identification. The baseline thus created was very 
high (94\%), resulting in a default decision to most frequent class 
(negative) for all samples. The K-Means clusterer was added to see if 
the number of pairs could be cut back, as a set of 50 documents already 
results in over 1200 possible pairs. K-Means was run with an iterative 
setting of $[$1:n-1$]$ clusters, after which the total of document 
clusters was counted. If two documents were never clustered together, 
they were stripped from the input for the neural network. This lead to a 
50\% reduction in document pairs, but also a 25\% reduction of correct 
pairs. Although this improved the baseline slightly, the data was still 
to biased for the neural network to be able to detect correct pairs in 
the training data, creating useless output of the network, which gave 
the Meanshift no additional features to work with. During development 
and tweaking of the neural network, the initial deadline for the shared 
task expired without submitting a working system. When the overview 
paper was published, the used method (on development set) could still be 
compared to the test set results of the other team. Based on only 
character based preprocessed data, as this lead to the best result in 
the PAN task of 2015 $[$2$]$, Meanshift output had a precision of 0.12 
on average. Using a Scikit-learn Countvectorizer for preprocessing the 
data with both word Ngram counts and normal word counts, the BCubed 
precision was 0.256, with a BCubed F-score of 0.376 (8$^{th}$ out of 
9 participants). This measure was also applied by the task committee and 
although the system was not submitted, results could be compared with 
other participants afterwards $[$3$]$. The second part of the task, the 
ranking of links within clusters resulted in a mean average precision of 
0.014 (5$^{th}$ place)\newpage


\newpage


\newcounter{numberedCntB}
\begin{enumerate}
\item \section{Introduction}\label{section:_Toc457563611}
\setcounter{numberedCntB}{\theenumi}
\end{enumerate}


With the publishing and sharing of documents being accessible to 
everybody, the need to verify what was written by who becomes apparent. 
Not only to prevent plagiarism, but also to prevent texts from being 
attributed to an author they do not belong to. Since 2011, PAN contains 
a shared task regarding automatic authorship identification. Where in 
recent years the task focused on determining whether or not a certain 
document belongs to a set of known documents of an author, the 2016 task 
is to cluster documents per author, without knowing the number of 
contributing authors. Additionally, the task also comprises a second 
step, in which the certainty of links has to be established between the 
different documents within a cluster/author. This second step is 
comparable with the earlier shared tasks, as it is a one on one 
comparison of documents. Depending on the similarity of two documents, 
the certainty of the author of both can be established (!!!Unresolved 
reference!!!). If the results of this shared task are satisfactory, the 
method can be applied on, for example, a portfolio of documents of 
students, to see if the author of all documents is the same. To make 
sure both steps are executed properly and no work is duplicated, the 
following research question will be the foundation of this research:

"\,Can a recurrent neural network help traditional clustering algorithms 
in clustering documents per author?"

To back up the main questions, the following sub questions can be 
formulated: 

-\ \ \ \ What features are important for the initial clustering per 
author?

-\ \ \ \ What features are important for establishing links between 
documents?

-\ \ \ \ Can the system be prevented from clustering based on obvious 
but wrong patterns such as topic?

In this thesis, at first related work and literature will be explored, 
based on which the approach to tackle the problem and answering the 
research question will be explained, followed by the results and the 
conclusion based on those results.

\begin{figure}[h]
\centering
\includegraphics[width=16.00cm,height=3.42cm]{media/image1.eps}
\end{figure}


Figure 1: Examples of complete clustering (left) and authorship-link 
ranking (right). $[$3$]$

\newpage


\begin{enumerate}
\setcounter{enumi}{\thenumberedCntB}
\item \section{Method}\label{section:_Toc457563612}
\setcounter{numberedCntB}{\theenumi}
\end{enumerate}


Plagiarism detection is a hot topic in scientific research, as it is now 
easier than ever to copy work someone else did and claim the results for 
yourself. Detectors are available for a long time, with plenty of 
research done on best practices and approaches, depending on the 
structure and the type of the available data. For this task, the data is 
split into 18 problems. Every problem has 50-100 documents that all have 
the same genre (news or review) and language (English, Dutch or Greek), 
equally distributed over the problems (!!!Unresolved reference!!!). 
Every problem also has a r of either 0.5, 0.7 or 0.9. R in indicative 
for the number of clusters with multiple authors; the lower r, the 
higher the chance that the problem has multi-author clusters. R 
correlates with max C, the maximum number of documents in a cluster. 

The small amount of data excludes the most promising solutions, as 
artificial neural networks combined with word embeddings are state of 
the art in solving practically any language related problem. Therefore, 
a less explored approach will be sought in this chapter.

Table 1: Evaluation datasets (left=training, right=test).

\begin{table}[h]
\centering
\begin{tabular}{|l|l|l|l|l|l|l|l|l|l|l|l|l|l|l|l|l|l|l|}
\hline
\textbf{ID} & \textbf{Lang} & \textbf{Genre} & \textbf{r} & 
\textbf{N} & \textbf{k} & \textbf{Links} & \textbf{max C} & 
\textbf{Words} & & \textbf{ID} & \textbf{Lang} & \textbf{Genre} 
& \textbf{r} & \textbf{N} & \textbf{k} & \textbf{Links} & 
\textbf{max C} & \textbf{Words} \\
\hline
001 & English & articles & 0.70 & 50 & 35 & 26 & 5 & 752.3 & & 001 & 
English & articles & 0.71 & 70 & 50 & 33 & 5 & 582.4 \\
\hline
002 & English & articles & 0.50 & 50 & 25 & 75 & 9 & 756.2 & & 002 & 
English & articles & 0.50 & 70 & 35 & 113 & 8 & 587.3 \\
\hline
003 & English & articles & 0.86 & 50 & 43 & 8 & 3 & 744.7 & & 003 & 
English & articles & 0.91 & 70 & 64 & 7 & 3 & 579.8 \\
\hline
004 & English & reviews & 0.69 & 80 & 55 & 36 & 4 & 977.8 & & 004 & 
English & reviews & 0.73 & 80 & 58 & 30 & 4 & 1,011.2 \\
\hline
005 & English & reviews & 0.88 & 80 & 70 & 12 & 3 & 1,089.7 & & 005 & 
English & reviews & 0.90 & 80 & 72 & 10 & 3 & 1,030.4 \\
\hline
006 & English & reviews & 0.50 & 80 & 40 & 65 & 5 & 1,029.4 & & 006 & 
English & reviews & 0.53 & 80 & 42 & 68 & 5 & 1,003.7 \\
\hline
007 & Dutch & articles & 0.89 & 57 & 51 & 7 & 3 & 1,074.7 & & 007 & 
Dutch & articles & 0.74 & 57 & 42 & 24 & 4 & 1,172.1 \\
\hline
008 & Dutch & articles & 0.49 & 57 & 28 & 76 & 7 & 1,321.9 & & 008 & 
Dutch & articles & 0.88 & 57 & 50 & 8 & 3 & 1,178.4 \\
\hline
009 & Dutch & articles & 0.70 & 57 & 40 & 30 & 4 & 1,014.8 & & 009 & 
Dutch & articles & 0.53 & 57 & 30 & 65 & 7 & 945.2 \\
\hline
010 & Dutch & reviews & 0.54 & 100 & 54 & 77 & 4 & 128.2 & & 010 & Dutch 
& reviews & 0.88 & 100 & 88 & 16 & 4 & 151.7 \\
\hline
011 & Dutch & reviews & 0.67 & 100 & 67 & 46 & 4 & 134.9 & & 011 & Dutch 
& reviews & 0.51 & 100 & 51 & 76 & 4 & 150.3 \\
\hline
012 & Dutch & reviews & 0.91 & 100 & 91 & 10 & 3 & 125.3 & & 012 & Dutch 
& reviews & 0.71 & 100 & 71 & 37 & 4 & 155.9 \\
\hline
013 & Greek & articles & 0.51 & 55 & 28 & 38 & 4 & 748.9 & & 013 & Greek 
& articles & 0.71 & 70 & 50 & 24 & 4 & 720.5 \\
\hline
014 & Greek & articles & 0.69 & 55 & 38 & 25 & 5 & 741.6 & & 014 & Greek 
& articles & 0.50 & 70 & 35 & 52 & 4 & 750.3 \\
\hline
015 & Greek & articles & 0.87 & 55 & 48 & 8 & 3 & 726.8 & & 015 & Greek 
& articles & 0.89 & 70 & 62 & 9 & 3 & 737.6 \\
\hline
016 & Greek & reviews & 0.91 & 55 & 50 & 6 & 3 & 523.4 & & 016 & Greek & 
reviews & 0.73 & 70 & 51 & 24 & 4 & 434.8 \\
\hline
017 & Greek & reviews & 0.51 & 55 & 28 & 55 & 8 & 633.9 & & 017 & Greek 
& reviews & 0.91 & 70 & 64 & 7 & 3 & 428.0 \\
\hline
018 & Greek & reviews & 0.73 & 55 & 40 & 19 & 3 & 562.9 & & 018 & Greek 
& reviews & 0.53 & 70 & 37 & 44 & 4 & 536.9 \\
\hline
\end{tabular}
\end{table}


\newpage


\begin{enumerate}
\item \subsection{Related work\\
}\label{section:_Toc457563613}
\end{enumerate}
Most recent work focused on comparing a new document to a set of known 
documents from one author, as that is the most logical approach if there 
is data available. The decision making was binary which does not make it 
a good source for this research. Instead, the method used by the winning 
team of PAN 2015 offers an interesting view on the problem. Using multi 
headed artificial neural networks$[$4$]$, he was able to link two 
documents together if they belonged to the same author. Using this 
approach for clustering would of course take a lot of processing time as 
all possible pairs must be compared, but if it works it might be very 
good at clustering all the right documents together. \\
Although primarily applied on other topics, clustering itself is of 
course well explored. Many algorithms have been developed that can find 
the center of dense clusters and compute to what range the cluster 
extends. In most cases however, the algorithm needs to know on 
beforehand how many clusters it is supposed to use to be able to find 
the right centers. An exception on this is Meanshift $[$5$]$, a hill 
climbing algorithm that keeps looking for better centroids and is able 
to add more clusters if necessary. Meanshift is proven to be effective 
in determining the number of clusters and has several software 
implementations that can be used off the shelf. Before Meanshift was 
developed, the research of Holmes and Forsyth already tried to achieve a 
very similar result $[$6$]$. They tested their method on the very famous 
(and notorious in NLP tasks) federalist papers, a set of articles about 
the American constitution written by three authors. Their dataset is 
comparable to the one used for this research, which makes their approach 
relevant. Many of the features used by Holmes and Forsyth are quite 
common these days, such as word frequency counting, like tf-idf now, and 
trying to find stylistic patterns for authors in documents. This last 
feature is especially important for this task; most clustering 
algorithms will look for word or ngram similiarity when trying to find 
similar documents. Similar words still can be an indication for similar 
documents, but there is a risk that although the documents are alike, 
they are not from the same author because they all have the same genre. 
A field in which stylistic features are even more important is 
engineering, where many articles are written on the same topic and 
genre. This makes the research of Berry and Sazonov about the clustering 
of technical documents an interesting and reliable source. The nature of 
the documents in their dataset makes them highly structured and 
restricted $[$7$]$, making the effect of stylistic feature selection 
extra visible. Although less technical, the documents in the current 
dataset can also be identical in structure and covered subjects. Berry 
and Sazonov say that sometimes the preference of an author for one word 
over another can be enough to distinguish who wrote what. This high 
influence of small features is something to keep in mind in this 
approach, as it can change the outcome in a very strong way.

\newpage


\begin{enumerate}
\item \subsection{Approach}\label{section:_Toc457563614}
\end{enumerate}


Based on the earlier work and proven concepts, a combined approach of 
all will be attempted. The largest restriction is that the entire system 
will have to be built in Python, as both the thesis and the shared task 
are on a tight schedule. This leaves no room for learning a new 
programming language. Python does provide all necessary tools for the 
task, and has modules for all desired features. Scikit-learn provides an 
excellent API for several clustering algorithms and preprocessing steps, 
and Keras allows for building an artificial neural network on either the 
TensorFlow or the Theano backend.

In order to train the system, the most promising features must be 
selected and applied. As the approach will be based on Bagnall's, his 
feature selection also applies for this problem. Training a neural 
network requires vast amounts of data, so the only way to do this is by 
looking at characters instead of words. On character level you can apply 
almost all techniques used on words, like the relative and absolute 
frequency of characters per document and the full corpus (tf-idf). 
Metadata about the documents will also be added, informing the system 
about the average sentence and word length, punctuation usage and number 
of mid-sentence capital letters. These stylistic features can be very 
indicative about an author, but according to Bagnall they should not be 
fed raw into the neural network. This could lead to the network 
assigning a too great weight to a small feature, and negatively 
influence decision making. Therefore, Bagnall proposes to normalize all 
uncommon characters $[$4$]$. Different commas, ellipses (\ldots ), 
quotation marks (single and double) and dashes (longer and shorter) all 
should be converted to a single style. Furthermore, additional 
whitespace must be stripped and all numbers and Latin characters in 
Greek texts should be normalized to a common placeholder to keep their 
weight evenly distributed. As final step, Bagnall recommends to convert 
every character into the NFKD unicode normal form. This form describes 
the character instead of displaying it, splitting it up if it has an 
accent on it to describe the accent separately (!!!Unresolved 
reference!!!). 

\begin{figure}[h]
\centering
\includegraphics[width=9.36cm,height=4.40cm]{media/image2.eps}
\end{figure}


Figure 2: Different unicode forms and their output.

The result of all normalization steps is a human unreadable list of 
strings per document, that can be used by both Keras an Scikit-learn for 
further preprocessing the data.\\
The assumption is that clustering and an artificial neural network can 
inform each other in order to achieve better results on the data. The 
idea of Bagnall to use a neural network for ranking document similarity 
is very promising, but time consuming on this many document pairs. Each 
of the 18 problems has at least 50 documents, which leads to at least 
1225 unique pairs per problem that need to be processed. It would 
therefore be very useful to remove certain pairs that are highly 
unlikely from the initial set. This removal should be done discreetly, 
as it is better to remove too less faulty pairs than too much pairs that 
should be together. This will make the training data less skewed, and 
cut back the processing time as well. The pairs can be shifted using a 
K-Means clusterer with an unspecified number of clusters. K-Means 
expects the user to set the desired number of clusters (k), so by 
iterating through a K of $[$1:n-1$]$, all possibilities will be tried. 
The cluster output can then be added together, to see which documents 
are never clustered together. That particular pair can then be removed 
from the full set. Once the set of document pairs has been trimmed, the 
remaining texts can be preprocessed to be fed into Keras's Long 
Short-Term Memory, or LSTM, neural network $[$8$]$. LSTM is a type of 
recurrent neural network that is especially good in processing and 
predicting texts, as it looks back at everything it learned until now. 
This means that an LSTM is able to learn rules from a correct pair and 
apply them on the current pair, even if there is a high number of 
incorrect pairs in between. Given the skewed data for this task, it is 
important that the network remembers the sparse correct pairs as good 
and as long as possible. LSTM's do have limitations on length however, 
so it might be that the long sequences of characters are too much for it 
to keep learning correctly. But just like any other implementation in 
Keras, LSTM's are stackable and should be able to fit the entire 
sequence of characters in the documents. LSTM expects the data in three 
dimensions of (nb\_sequences, nb\_samples, input\_dim). Sequences is 
defined by the total number of sequences, samples by the length of one 
document and input dimension by the total of different characters in the 
vocabulary, so 26 + some special ones. The data itself must be one-hot 
encoded, with each character being represented by a number in 
range$[$input\_dim$]$. Once encoded, the entire dataset will be 
converted into a 3D matrix by Keras's preprocessing tools for the 
network to use. As an extra feature, character embeddings will be 
constructed using skipgrams $[$1$]$. By encoding all possible character 
pairs with either 1 if they are neighbors, and 0 if the pairs are far 
apart, a vector can be built of what characters are likely and unlikely 
to occur together. \\
The ratings of one document versus every other will be used as an 
additional feature for the Meanshift algorithm. Experiments will have to 
point out whether the feature should be shaped like a list with 1's and 
0's, or more like a dictionary with one document as key and all its 
matches as values. The Meanshift implementation in Scikit-learn offers 
few parameters, and is able to calculate the bandwidth it should use 
based on the data. A too small bandwidth results in many clusters while 
there might be overlapping ones, and a too large bandwidth merges too 
many clusters, resulting in only a few final clusters. The output of 
Meanshift will be a list of documents per cluster, that can be 
transformed in the same JSON format the task committee provides the 
truth data in. Using the online review environment Tira $[$9$]$, 
automatic evaluation of the answers given versus the gold standard data 
will be done. Evaluation results will be according to the Bcubed score 
$[$10$]$, a measure that combines the scores within a cluster with the 
ones across clusters for computing precision and recall (!!!Unresolved 
reference!!!). This results in different scores per language and per 
genre, with a total score over all problems that all will be described 
in the next chapter. 

\begin{figure}[h]
\centering
\includegraphics[width=10.13cm,height=4.11cm]{media/image3.eps}
\end{figure}


Figure 3: Calculation of Bcubed precision and recall

\begin{enumerate}
\setcounter{enumi}{\thenumberedCntB}
\item \section{Results}\label{section:_Ref457304642}
\setcounter{numberedCntB}{\theenumi}
\end{enumerate}


When coding the software based on the dataflow from the previous 
chapter, it soon turned out that the sequences of characters were in 
fact too long for a LSTM recurrent neural network to process. The 
correct pairs were just too sparse for the network to remember the 
previous one when encountering a new one, resulting in a preference for 
the most frequent class with no apparent learning. This was only 
worsened by using the pairwise comparison in the neural network. With 
the aforementioned example of 1200 pairs, created from just 50 
documents, only about 70 of 1200 would be correct pairs. This gives a 
tough to beat baseline of 94\%, with far too little correct data for the 
network to learn from. Using the K-Means clusterer to strip unlikely 
pairs was successful, removing about half of the total, but at a cost of 
25\% of correct pairs. In absolute numbers it was a very successful 
method, but relatively without effect to lower the most frequent class 
baseline. 

Meanwhile, during the development of the system and continuous tweaks to 
get the neural network going, the deadline for the shared task expired. 
No working software was submitted because the pipeline could not be 
completed in time, which made official evaluation at that point 
impossible too. Unofficial measures show a precision of 0.1 per problem 
on average over the 20 most common pairs in a cluster. This means that, 
of all clusters returned from the K-Means iterations, only the top 20 is 
taken into account. Of this 20, only 10\% (so 2 pairs) would usually be 
correct according to the gold standard. So K-Means was an outstanding 
preprocessor if executed before the neural network, but performed bad if 
its output was used as final data. The iteration over all possible 
number of clusters made the output unfit for comparison with the gold 
standard, so a single K should be chosen or computed first. The returned 
data after iteration gave no indication when it performed best, so only 
a rule based K could be implemented. Exploring the development data 
showed that in most problems less than one third of all articles share 
clusters, so two third of all clusters are only populated by a single 
document or author. Implementing K-Means with K=2/3(N) worked out well 
in F-score (!!!Unresolved reference!!!), it is a highly unstable method 
on unseen testdata, as it assumes that the shape will always be the same 
as that of the development data. The good results in !!!Unresolved 
reference!!! are only because of rules based on the observations, and 
working with a rule based system is never recommended if sufficient data 
for machine learning is available. Despite performing significantly less 
than K-Means, the idea was that Meanshift is better prepared for unseen 
data, making it a better candidate for final implementation.

With the neural network out of the system, a different set of features 
was selected to optimize the clustering algorithms. The data provided 
was not very extensive, but big enough for word based features for 
clustering. Changing from characters to words greatly influenced 
clustering performance, with acceptable results. However, adding 
features outside of stripping accents and ngrams did not contribute to a 
higher score. Clustering performed the same, regardless of adding word 
or punctuation counts. The system seemed to ignore topic similar 
documents by itself, clustering based on stylistic features instead in 
many cases. Once again, this was primarily the case with K-Means that 
performed much better than expected.









Table 2: Experiment results per system

\begin{table}[h]
\centering
\begin{tabular}{|l|l|l|l|l|l|l|l|}
\hline
\textbf{System } & Complete clustering & & Link ranking & Runtime (s) 
\\
\hline
� & B3 F & B3 rec. & B3 prec. & � & MAP & � \\
\hline
MS Word & 0.376 & 0.877 & 0.256 & & 0.014 & 208 \\
\hline
MS Char & 0.065 & 1 & 0.034 & & 0.020 & 174 \\
\hline
KM Word & 0.696 & 0.732 & 0.676 & & 0.008 & 191 \\
\hline
KM Char & 0.065 & 1 & 0.034 & � & 0.020 & 135 \\
\hline
Baseline & 0.811 & 0.697 & 1 & � & 0 & 120 \\
\hline
\end{tabular}
\end{table}


Because no ranking within clusters could be made without a neural 
network, it was decided to give every link of two documents within a 
cluster a certainty of 1, trusting on the accuracy of Meanshift. Due to 
the presentation of the overview paper $[$3$]$ of the 2016 shared task, 
the evaluation script and other results also were made available for 
comparison (!!!Unresolved reference!!!). Note that the results in the 
table are presumably on the test set, while our results are on the 
development set. As seen before the two datasets are very alike 
(!!!Unresolved reference!!!), so the results also might be comparable, 
but this should be said with great caution. 

\\
The results of Meanshift are surprisingly good for the simple 
implementation via Scikit-learn, although only Bcubed recall for 
clustering and mean average precision for ranking are above the random 
baseline. Especially Singleton baseline performed very well on 
clustering, as the large majority of clusters only consisted of a single 
document. Only Bagnall, also winner of last year and following a similar 
approach as this research, and Kocher were able to beat it in Bcubed 
F-score, and only just. Cosine baseline on the other hand seemed very 
low and easy to beat at first sight, because it would purely focus on 
word similarities between documents. This feature was exactly what all 
participants wanted to avoid as it would rank by topic instead of by 
author, but even by actively only selecting stylometric features only 
two participants were able to beat it.

Table 3: FInal results of the PAN 2016 shared task for author 
clustering. $[$3$]$

\begin{table}[h]
\centering
\begin{tabular}{|l|l|l|l|l|l|l|l|l|}
\hline
\textbf{Participant } & \textbf{ Complete clustering } & & \textbf{ 
Authorship-link ranking} & \textbf{Runtime} \\
\hline
� & B3 F & B3 rec. & B3 prec. & & MAP & RP & P@10 & � \\
\hline
Bagnall & \textbf{0.822} & 0.726 & 0.977 & & \textbf{0.169} & 
\textbf{0.168} & \textbf{0.283} & 63:03:59 \\
\hline
Gobeill & 0.706 & 0.767 & 0.737 & & 0.115 & 0.131 & 0.233 & 00:00:39 \\
\hline
Kocher & \textbf{0.822} & 0.722 & \textbf{0.982} & & 0.054 & 0.050 & 
0.117 & 00:01:51 \\
\hline
Kuttichira & 0.588 & 0.720 & 0.512 & & 0.001 & 0.010 & 0.006 & 00:00:42 
\\
\hline
\underline{Louwaars MS} & \underline{0.376} & \underline{0.877} & 
\underline{0.256} & & \underline{0.014} & \underline{-} & 
\underline{-} & \underline{00:03:28} \\
\hline
Mansoorizadeh \textit{et al.} & 0.401 & 0.822 & 0.280 & & 0.009 & 
0.012 & 0.011 & 00:00:17 \\
\hline
Sari \& Stevenson & 0.795 & 0.733 & 0.893 & & 0.040 & 0.065 & 0.217 & 
00:07:48 \\
\hline
Vartapetiance \& Gillam & 0.234 & \textbf{0.935} & 0.195 & & 0.012 & 
0.023 & 0.044 & 03:03:13 \\
\hline
Zmiycharov \textit{et al.} & 0.768 & 0.716 & 0.852 & & 0.003 & 0.016 & 
0.033 & 01:22:56 \\
\hline
BASELINE-Random & 0.667 & 0.714 & 0.641 & & 0.002 & 0.009 & 0.013 & - \\
\hline
BASELINE-Singleton & 0.821 & 0.711 & \textbf{1.000} & & - & - & - & - 
\\
\hline
BASELINE-Cosine & - & - & - & & 0.060 & 0.074 & 0.139 & - \\
\hline
\end{tabular}
\end{table}


\newpage


Detailed results per genre and language for clustering can be found in 
!!!Unresolved reference!!!. Interesting in these results is the high 
difference in languages where most teams have an almost equal score for 
all three. Although the system was only working with character features, 
somehow it was much better in clustering Dutch texts than Greek or 
English. Also, the ratio between scores of articles and reviews differs 
per participant, with no clear pattern in which genre is easier or 
tougher to cluster. Contrary to most participants, a lower r only leads 
to higher F-scores. The difference between the participants and 
baselines is even clearer in this table, and shows that despite the 
naivety of the baselines they perform far better than most teams. 

Table 4: Evaluation results (mean BCubed F-score) for the complete 
author clustering task. $[$3$]$

\begin{table}[h]
\centering
\begin{tabular}{|l|l|l|l|l|l|l|l|l|l|}
\hline
\textbf{Participant} & \textbf{Overall} & \textbf{Articles} & 
\textbf{Reviews} & \textbf{English} & \textbf{Dutch} & \textbf{
Greek} & \textbf{\textit{r}}\textit{?}\textbf{0.9} & \textbf{
\textit{r}}\textit{?}\textbf{0.7} & \textbf{\textit{r}}
\textit{?}\textbf{0.5} \\
\hline
Bagnall & \textbf{0.822} & \textbf{0.817} & \textbf{0.828} & 
\textbf{0.820} & \textbf{0.815} & 0.832 & 0.931 & 0.840 & \textbf{
0.695} \\
\hline
Kocher & \textbf{0.822} & \textbf{0.817} & 0.827 & 0.818 & \textbf{
0.815} & \textbf{0.833} & \textbf{0.933} & \textbf{0.843} & 0.690 
\\
\hline
BASELINE-Singleton & 0.821 & \textbf{0.819} & 0.823 & \textbf{0.822} 
& \textbf{0.819} & 0.822 & \textbf{0.945} & 0.838 & 0.680 \\
\hline
Sari \& Stevenson & 0.795 & 0.789 & 0.801 & 0.784 & 0.789 & 0.813 & 
0.887 & 0.812 & 0.687 \\
\hline
Zmiycharov \textit{et al.} & 0.768 & 0.761 & 0.776 & 0.781 & 0.759 & 
0.765 & 0.877 & 0.777 & 0.651 \\
\hline
Gobeill & 0.706 & 0.800 & 0.611 & 0.805 & 0.606 & 0.707 & 0.756 & 0.722 
& 0.639 \\
\hline
BASELINE-Random & 0.667 & 0.666 & 0.667 & 0.668 & 0.665 & 0.667 & 0.745 
& 0.678 & 0.577 \\
\hline
Kuttichira & 0.588 & 0.626 & 0.550 & 0.579 & 0.584 & 0.601 & 0.647 & 
0.599 & 0.519 \\
\hline
Mansoorizadeh \textit{et al.} & 0.401 & 0.367 & 0.435 & 0.486 & 0.256 
& 0.460 & 0.426 & 0.373 & 0.403 \\
\hline
\underline{Louwaars MS} & \underline{0.376} & \underline{0.386} & 
\underline{0.367} & \underline{0.27} & \underline{0.465} & 
\underline{0.394} & \underline{0.356} & \underline{0.376} & 
\underline{0.397} \\
\hline
Vartapetiance \& Gillam & 0.234 & 0.284 & 0.183 & 0.057 & 0.595 & 0.049 
& 0.230 & 0.241 & 0.230 \\
\hline
\end{tabular}
\end{table}


!!!Unresolved reference!!! makes the results for authorship ranking 
insightful per genre and language. Again, no genre is consequently 
ranked better than the other, but there seems to be a tendency of Dutch 
ranks scoring the lowest throughout all teams. Although the achieved 
mean average precision of 0.014 seems incredibly low, in this table it 
turns out to be halfway between the best and worst results. The fully 
random baseline was narrowly beaten, but cosine similarity appeared to 
be far more indicative of link ranks than assumed, especially for Greek. 
!!!Unresolved reference!!! also shows the same rising scores as 
!!!Unresolved reference!!! with a descending r, but for the ranking all 
other teams show the same behavior. 

Table 5: Evaluation results (MAP) for the authorship-link ranking task. 
$[$3$]$

\begin{table}[h]
\centering
\begin{tabular}{|l|l|l|l|l|l|l|l|l|l|l|}
\hline
\textbf{Participant} & \textbf{Overall} & \textbf{Articles} & 
\textbf{Reviews} & \textbf{English} & \textbf{Dutch} & \textbf{
Greek} & \textbf{\textit{r}}\textit{?}\textbf{0.9} & \textbf{
\textit{r}}\textit{?}\textbf{0.7} & \textbf{\textit{r}}
\textit{?}\textbf{0.5} \\
\hline
Bagnall & \textbf{0.169} & \textbf{0.174} & \textbf{0.163} & 
\textbf{0.126} & \textbf{0.109} & \textbf{0.272} & \textbf{0.064
} & \textbf{0.186} & \textbf{0.257} \\
\hline
Gobeill & 0.115 & 0.119 & 0.110 & 0.097 & 0.079 & 0.168 & 0.040 & 0.105 
& 0.198 \\
\hline
BASELINE-Cosine & 0.060 & 0.063 & 0.057 & 0.053 & 0.053 & 0.074 & 0.019 
& 0.054 & 0.107 \\
\hline
Kocher & 0.054 & 0.047 & 0.061 & 0.032 & 0.044 & 0.085 & 0.042 & 0.058 & 
0.063 \\
\hline
Sari \& Stevenson & 0.040 & 0.033 & 0.047 & 0.009 & 0.042 & 0.069 & 
0.017 & 0.041 & 0.062 \\
\hline
\underline{Louwaars MS} & \underline{0.014} & & \underline{0.015} 
& \underline{0.013} & \underline{0.016} & \underline{0.007} & 
\underline{0.018} & \underline{0.004} & \underline{0.01} & 
\underline{0.027} \\
\hline
Vartapetiance \& Gillam & 0.012 & 0.010 & 0.014 & 0.014 & 0.006 & 0.016 
& 0.010 & 0.008 & 0.017 \\
\hline
Mansoorizadeh \textit{et al.} & 0.009 & 0.013 & 0.004 & 0.006 & 0.010 
& 0.010 & 0.002 & 0.009 & 0.014 \\
\hline
Zmiycharov \textit{et al.} & 0.003 & 0.002 & 0.004 & 0.001 & 0.000 & 
0.009 & 0.002 & 0.003 & 0.004 \\
\hline
BASELINE-Random & 0.002 & 0.002 & 0.001 & 0.001 & 0.002 & 0.002 & 0.001 
& 0.001 & 0.002 \\
\hline
Kuttichira & 0.001 & 0.002 & 0.001 & 0.001 & 0.002 & 0.001 & 0.001 & 
0.002 & 0.001 \\
\hline
\end{tabular}
\end{table}
\newpage


\begin{enumerate}
\setcounter{enumi}{\thenumberedCntB}
\item \section{Discussion}\label{section:_Toc457563616}
\setcounter{numberedCntB}{\theenumi}
\end{enumerate}
In this final chapter all unexpected results will be presented, with a 
possible explanation if there is one at hand. Also, a general reflection 
on the designing and development of the system will be given with an 
answer on the research questions.

\begin{enumerate}
\item \subsection{Evaluation}\label{section:_Toc457563617}
\end{enumerate}
The results presented in chapter \ref{section:_Ref457304642}. are far 
worse than anticipated on beforehand, but seeing them in context with 
the other teams helps understanding that they are not that bad. Given 
the failure of the most important processing part of the system, the 
recurrent neural network, the result is quite satisfactory and in line 
with the others. One thing that stands out however, are the 
irregularities in scores in comparison with other teams. Where most 
teams have an equal score for all languages, our system has a clear 
preference for one over the other, and is especially good in Dutch and 
bad in English. An explanation for this can be given by the unicode 
normalization step in the process. unicode encodes accents on characters 
as separate characters, and as Dutch has more accented letters than 
English, these encodings might be highly informative. 

R values are also the reverse of most teams, with better results with a 
lower r. This can be explained by the conservatism of Meanshift, as it 
tends to use as few clusters as possible, and therefore performs better 
in problems with a small k. Finding a way to increase the number 
clusters found by Meanshift would therefore increase the overall 
performance of the software and seems promising for future research. 
Another recommendation is to apply cosine similarity in link ranking, 
because it worked surprisingly well for cosine baseline. At first the 
thought was that the baseline would be misguided by topic of documents, 
as even the task committee underwrote, but this turned out to be false 
as the baseline had the third score of all teams. An indication of the 
similarity preferences of the system was already given in the clustering 
process, as clustering did not improve with adding stylistic features. 
Using cosine similarity thus would be a good first step in ranking, 
combined with a more sophisticated way of ranking documents based on 
similarity.

\begin{enumerate}
\item \subsection{Conclusion}\label{section:_Toc457563618}
\end{enumerate}
Looking back at working on this shared task, it really is a pity that no 
system was ready for submitting at the time of the deadline. With no 
submission the opportunity of competing with other teams was gone, 
making the development and outcome less exciting. Still, the achieved 
results are not too bad given shape of the final system. The proposed 
approach with two clustering methods as input and output of a recurrent 
neural network seemed very promising based on earlier results. This is 
underlined by the fact that Bagnall's system of this year is based on 
his submission and suggestions of last year. Unfortunately, the shared 
task's individual papers are not released yet, so it is not possible to 
see how Bagnall got his system to work so well, and what steps were 
missed in our software. His results do show however that it is possible 
and useful to use recurrent neural networks for author clustering and 
link ranking, but this cannot be used to answer the research question 
here. 

The answer to "\,Can a recurrent neural network help traditional 
clustering algorithms in clustering documents per author?" according to 
this research is negative, as the approach originally proposed could not 
be implemented. The resulting neural network did not help at all to 
inform the clustering or to rank the links in the output. An interesting 
finding though, is that a K-Means iterative approach for stripping away 
non-likely pairs is working well. By using this preprocessing step, half 
of all pairs fed to the neural network could be removed, greatly 
decreasing processing time. As this time is the biggest downside of the 
winning team, cutting it back by half would be a great win. 

\newpage


\section{Bibliography}\label{section:_Toc457563619}
1.\ \ \ \ Mikolov, T., et al., \textit{Efficient estimation of word 
representations in vector space.} arXiv preprint arXiv:1301.3781, 2013.

2.\ \ \ \ Stamatatos, E., et al., \textit{Overview of the Author 
Identification Task at PAN 2015.}

3.\ \ \ \ Stamatatos, E., et al., \textit{Clustering by Authorship 
Within and Across Documents.} CEUR Workshop Proceedings, 2016. \textbf{
Working Notes Papers of the CLEF 2016 Evaluation Labs}.

4.\ \ \ \ Bagnall, D., \textit{Author identification using multi-headed 
recurrent neural networks.} arXiv preprint arXiv:1506.04891, 2015.

5.\ \ \ \ Comaniciu, D. and P. Meer, \textit{Mean shift: A robust 
approach toward feature space analysis.} IEEE Transactions on pattern 
analysis and machine intelligence, 2002. \textbf{24}(5): p. 603-619.

6.\ \ \ \ Holmes, D.I. and R.S. Forsyth, \textit{The Federalist 
revisited: New directions in authorship attribution.} Literary and 
Linguistic Computing, 1995. \textbf{10}(2): p. 111-127.

7.\ \ \ \ Berry, D. and E. Sazonov. \textit{Clustering technical 
documents by stylistic features for authorship analysis}. in \textit{
SoutheastCon 2015}. 2015. IEEE.

8.\ \ \ \ Hochreiter, S. and J. Schmidhuber, \textit{Long short-term 
memory.} Neural computation, 1997. \textbf{9}(8): p. 1735-1780.

9.\ \ \ \ Potthast, M., et al. \textit{Improving the Reproducibility of 
PAN's Shared Tasks}. in \textit{International Conference of the 
Cross-Language Evaluation Forum for European Languages}. 2014. 
Springer.

10.\ \ \ \ Rosales-M�ndez, H. and Y. Ram�rez-Cruz. \textit{CICE-BCubed: 
A new evaluation measure for overlapping clustering algorithms}. in 
\textit{Iberoamerican Congress on Pattern Recognition}. 2013. 
Springer.



\end{document}
